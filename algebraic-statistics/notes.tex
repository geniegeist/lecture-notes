\documentclass[a4paper, 11pt]{article}
\usepackage{marginnote}


\def\nterm {}
\def\nyear {}
\def\nlecturer {}
\def\ncourse {Algebraic Statistics}

\ifx \nlecturer\undefined 
	\author{Notes taken by Viet Duc Nguyen}
\else
	\author{Based on lectures by \nlecturer}
\fi
\date{\nterm\ \nyear}
\title{\ncourse}

\usepackage[utf8]{inputenc}
\usepackage{amsmath,amsthm,amssymb}
\makeatletter
\def\th@plain{%
  \thm@notefont{}% same as heading font
  \itshape % body font
}
\def\th@definition{%
  \thm@notefont{}% same as heading font
  \normalfont % body font
}
\makeatother
\usepackage{mathtools}
\usepackage{bbm}
\usepackage{marvosym}
\usepackage{array}
\usepackage{geometry}
\usepackage[toc,titletoc,title]{appendix}
\usepackage[hidelinks]{hyperref}
\usepackage{framed}
\usepackage{caption}
\usepackage{subcaption}
\usepackage{enumitem}
\usepackage{tikz}
\usetikzlibrary{patterns}
\usetikzlibrary{shapes}
\usetikzlibrary{plotmarks}
\usetikzlibrary{cd}

\usepackage{float}
\usepackage{xcolor}
\hypersetup{
	colorlinks,
	linkcolor={red!50!black},
	citecolor={red!50!black},
	urlcolor={red!50!black}
}


\makeatletter
\let\@real@maketitle\maketitle
\renewcommand{\maketitle}{\@real@maketitle\begin{center}\begin{minipage}[c]{0.9\textwidth}\centering\footnotesize Updated on \date{\today}
 \\	These notes are not endorsed by the lecturer.\end{minipage}\end{center}}
\makeatother

\DeclareMathOperator*{\argmax}{arg\,max}
\DeclareMathOperator*{\argmin}{arg\,min}


\theoremstyle{definition}
\newtheorem{thm}{Theorem}[section]
\newtheorem{lemma}[thm]{Lemma}
\newtheorem*{lemma*}{Lemma}
\newtheorem{cor}[thm]{Corollary}
\newtheorem{prop}[thm]{Proposition}
\newtheorem{defi}[thm]{Definition}
\newtheorem{eg}[thm]{Example}
\newtheorem{remark}[thm]{Remark}
\newtheorem{ex}[thm]{Exercise}
\newtheorem{note}[thm]{Note}

\newcommand{\PhantC}{\phantom{\colon}}%
\newcommand{\PhantSQ}{\phantom{\sqrt{\hspace{0.3ex}}}}%
\newcommand*{\vertbar}{\rule[-1ex]{0.5pt}{2.5ex}}
\newcommand*{\horzbar}{\rule[.5ex]{2.5ex}{0.5pt}}

% https://tex.stackexchange.com/questions/63355/wrapping-cmidrule-in-a-macro
\ExplSyntaxOn
\makeatletter
\newcommand{\CMidRule}{\noalign\bgroup\@CMidRule{}}
\NewDocumentCommand{\@CMidRule}{
    m % Material to reinsert before cmidrule.
    O{0.0ex} % #1 = left adjust
    O{0.0ex} % #1 = right adjust
    m  %       #3 = columns to span
}{
    \peek_meaning_remove_ignore_spaces:NTF \CMidRule
      { \@CMidRule { #1 \cmidrule[\cmidrulewidth](l{#2}r{#3}){#4} } }
      { \egroup #1 \cmidrule[\cmidrulewidth](l{#2}r{#3}){#4} }
}
\makeatother
\ExplSyntaxOff


\begin{document}
\maketitle
\tableofcontents

\section{Dimension Theory}

\begin{defi}[Affine Hilbert function]
  Let \( I \subset k[x_1,...,x_n] \) be an ideal. The \textbf{affine Hilbert function} of \( I \) is defined to be
  \begin{align*}
    \mathrm{aHF} _{R/I}:
    s \mapsto \mathrm{dim}\left( R_{\leq s} / I_{\leq s} \right).
  \end{align*}
\end{defi}

\begin{remark}[Finite dimensional vector space]
  Note that \( R_{\leq s} / I_{\leq s} \) is a subspace of the \( k \)-vector space \( R_{\leq s} \), the latter is a \emph{finite-dimensional} vector space since there exist \( {n + s  \choose s} \) monomials of degree \( \leq s \); these monomials form a \emph{basis}. So both the vector space and the subspace are finite dimensional and we can compute \( \mathrm{dim}\left( R_{\leq s} / I_{\leq s} \right) = \mathrm{dim}\left( R_{\leq s} \right) - \mathrm{dim}(I_{\leq s} ) \).
\end{remark}

For a monomial ideal we have an alternative interpretation of the affine Hilbert function: it counts the number of monomials not in the ideal.

\begin{prop}[Affine Hilbert function of monomial ideals]
  Let \( I \) be a monomial ideal. Then \( \mathrm{aHF}_{R / I}(s) \) is equivalent to the map 
  \begin{align*}
    \mathrm{aHF} _{R/I}:
    s \mapsto \text{  counts the number of monomial of degree \( \leq s \) not in \( I \)}.
  \end{align*}
\end{prop}

\begin{remark}
  If \( I \) is a monomial ideal, we know for sufficiently large \( s \) the above function can be represented by a polynomial, which we call the \textbf{Hilbert polynomial} \( \mathrm{aHP}_{R / I} \). Moreover, this polynomial is of degree \( \mathrm{dim}(V(I)) \), where by definition \( \mathrm{dim}(V(I)) \) is defined as the dimension of the largest coordinate subspace in \( V(I) \).
\end{remark}


\begin{prop}[Reduction to monomial ideals]
  For any graded order and any ideal \( I \), we have \( \mathrm{aHF}_{R / I} = \mathrm{aHF}_{R / (\mathrm{LT}(I))} \).
\end{prop}

This allows us to define the Hilbert polynomial  for arbitrary ideals. Just pick any graded order and define \( \mathrm{aHP} \) to be the polynomial representing \( \mathrm{aHF}_{R / \mathrm{LT}(I)} \).
\begin{align*}
  \mathrm{aHF}_{R / I} \coloneqq \mathrm{aHF}_{R / \mathrm{LT}(I)} = C(\mathrm{LT(I)}) = \text{Hilbert polynomial of \( \mathrm{LT}(I) \)}
\end{align*}

\begin{defi}[Affine Hilbert polynomial]
  Let \( I \) be an ideal in \( k[x_1,...,x_n] \). For sufficiently large \( s \), the polynomial \( \mathrm{aHP}_{R / I} \) that equals \( \mathrm{aHF}_{R / I} \) is called the \textbf{affine Hilbert polynomial}.
\end{defi}

As previously stated, the degree of the affine Hilbert polynomial equals the dimension of \( V(I) \) if \( I \) is a monomial ideal. 

\begin{defi}[Dimension of a variety]
  The \textbf{dimension of a variety} \( V \subset k^n \) is the degree of the affine Hilbert polynomial \( \mathrm{aHP}_{R / I(V)} \).
\end{defi}

We gave a purely algebraic description of the dimension of a variety:
\begin{align*}
  \text{dimension of a variety } = \text{ degree of a polynomial}
\end{align*}

\begin{remark}[Warning]
  Let \( V \) be any variety with \( V = V(I) \) for some ideal \( I \). Then the degree of the Hilbert polynomial of \( I \) need not be equal to the dimension of \( V \). This only holds for algebraically closed fields (if \( k = \bar k \), then the Nullstellensatz holds and \( I(V(I)) = \sqrt I \)).
  \begin{align*}
    I \text{ such that } V = V(I) \nRightarrow \mathrm{dim}(I) = \mathrm{dim}(V)
  \end{align*}
\end{remark}

\begin{prop}[Characterization of zero dimensional varieties]
  Let \( V \subset k[x_1,...,x_n] \) be a nonempty affine variety. Then 
  \begin{align*}
    |V| < \infty \iff \mathrm{dim}(V) = 0 .
  \end{align*}
\end{prop}

\begin{proof}
  If \( V  \) is empty, then the dimension is not defined. So assume \( V \neq \emptyset \).

  \begin{itemize}
    \item \( \implies \): Assume that \( V = \left\{ v_1,...,v_k \right\} \subset \mathbb R^n \). For each \( i=1,...,n \) we define the polynomial 
    \begin{align*}
      f_i(x) = (x_i - v_{1i})(x_i -  v_{2i}) \cdots (x_i - v_{ki}) \in I(V).
    \end{align*}
    Observe that \( \mathrm{LT}(f_i) = x_i^k \) for any graded order. So \( (\mathrm{LT}(I(V))) \) contains \( x_1^k, \dots , x_n^k \).

    By definition, 
    \begin{align*}
      \mathrm{dim}(V) = \mathrm{deg}(\mathrm{aHP}_{R / I(V)}) =  \mathrm{deg}(\mathrm{aHP}_{R / \mathrm{LT}(I(V))}).
    \end{align*}
    The degree of the Hilbert polynomial of a monomial ideal \( J \) equals the dimension of \( V(J) \) where the dimension of \( V(J) \) is defined to be the dimension of the largest coordinate subspace in \( V(J) \). Thus, by setting \( J = \mathrm{LT}(I(V)) \), we obtain
    \begin{align*}
      \mathrm{deg}(\mathrm{aHP}_{R / \mathrm{LT}(I(V))}) = \mathrm{dim}(V(\mathrm{LT}(I(V)))).
    \end{align*}
    Since \( \mathrm{LT}(I(V)) \) contains \( x_1^k, \dots , x_n^k  \), its vanishing ideal consists of points with \( x_1 = ... = x_n = 0 \). Hence, \( V(\mathrm{LT}(I(V))) = \left\{ 0 \right\} \). Clearly, \( \mathrm{dim}(\left\{ 0 \right\}) = 0\) (since any coordinate subspace of \( \left\{ 0 \right\} \) is of dimension \( 0 \)).

    \item \( \impliedby \): Let \( V \) be of dimension \( 0 \). Hence, the Hilbert polynomial of \( I(V) \) is a constant for sufficiently large \( s \). This means 
    \begin{align*}
      \mathrm{dim}(k[x_1,...,x_n]_{\leq s} / I(V)_{\leq s}) = C.
    \end{align*}
    Let \( s \geq C \). Then for any \( i=1, ..., n \) the set of vectors \( x_i^{\{0, \dots, s\}} \) is linearly dependent in \( k[x_1,...,x_n]_{\leq s} / I(V)_{\leq s} \). So, define the polynomial \( f_i \) to be
    \begin{align*}
      0 \neq f_i \coloneqq \sum_{k=0}^s \alpha_k x_i^k \in I(V)_{\leq s}.
    \end{align*}    
    Since this holds for any \( s \geq C \), \( f_i \neq 0 \) in \( I(V) \). Hence, \( f_i \in I(V) \) has only finitely many roots (since it is nonzero); also \( f_i \) vanishes on \( V \). Thus, \( V \) has only finitely elements \( y \in V \) with different coordinates \( y_i \). Since \( i \) was chosen arbitrarily, \( V \) is finite.
  \end{itemize}
\end{proof}





\section{Maximum Likelihood Estimation}

\begin{defi}[Parameter space]
  An open subset \( \Theta \subset \mathbb R^d \) is called the \textbf{parameter space}. Elements \( \theta = (\theta_1,...,\theta_d) \in \Theta \) are called \textbf{parameters}.
\end{defi}

\begin{defi}[Algebraic statistical model]
  An \textbf{algebraic statistical model} is a map \( \mathbf f = (f_1,...,f_m): \mathbb C^d \to \mathbb C^m \) with \(f_i \in \mathbb Q[\theta_1,...,\theta_d] \) such that 
  \begin{itemize}
    \item \( f_1 + ... + f_m - 1 = 0 \in \mathbb Q[\theta_1,...,\theta_d] \) is the zero polynomial, and 
    \item \( \mathbf f(\theta) > 0 \) for all parameters \( \theta \in \Theta \).
  \end{itemize}
  For each parameter \( \theta \in \Theta \) a statistical model \( \mathbf f \) defines a {\textbf{probability distribution}} on the state space \( \left\{ 1,...,m \right\} \), that is, \( f_i(\theta) = p_i \) means that state \( i \in \left\{ 1,...,m \right\} \) occurs with probability \( p_i \in [0,1] \) for parameter \( \theta \).
\end{defi}

Assume we are given the number of occurrences of states \( 1,...,m \) of an experiment by a vector \( \mathbf u = (u_1,...,u_m) \in \mathbb N^m \). Fix a parameter \( \theta \in \Theta \). The probability that the state \( i \in \left\{ 1,...,m \right\} \) appears \( u_i \) times is given by
\begin{align*}
  f_i(\theta)^{u_i}.
\end{align*}
The problem of \textbf{maximum likelihood estimation} is to find the best parameter \( \theta \) that maximizes \(   \prod_{i=1}^m f_i(\theta)^{p_i}\). Maximizing this function is equivalent to maximizing the so called \textbf{log-likelihood function}
\begin{align*}
  \ell_u(\theta) = \sum^m_{i=1}u_i \cdot \log{f_i(\theta)}.
\end{align*}
From calculus, we know that a necessary condition for a local and global maximum \( \hat \theta \) is that the derivative of \( \ell_u \) must vanish at \( \hat \theta \) (note that if \( \Theta \) were not open, then the derivative need not vanish at a global maximum; on the other hand a global maximum need no exist). Thus, we need to find a solution to \( d \)-many equations, called the \textbf{critical equations}
\begin{align*}
  \frac{\partial \ell_u}{\partial \theta_1} &= \sum^m_{i=1} \frac{u_i}{f_i} \frac{\partial f_i}{\partial \theta_1} = 0 \\
  &... \\
  \frac{\partial \ell_u}{\partial \theta_d} &= \sum^m_{i=1} \frac{u_i}{f_i} \frac{\partial f_i}{\partial \theta_d} = 0 \\
\end{align*}

\begin{mdframed}
\begin{center}
  \textbf{{Our goal is to find all solutions \( \theta \in \mathbb C^d \) to the critical equations.}}
\end{center}
\end{mdframed}

Let \( \mathcal H \) be the locus where all the denominators of the rational functions in the critical equations vanish. The set of solutions \( \theta \in \Theta \) outside \( \mathcal H \) is an \emph{algebraic variety} in \( \mathbb C^d \) called the \textbf{likelihood variety}.

\begin{prop}
  For generic data \( u \), the number of solutions to the critical equations is independent of \( u \).
\end{prop}

\begin{proof}
  \begin{align*}
    \frac{\partial}{\partial \theta_i} \log{\frac{f_j}{g_j}} = \frac{g_j}{f_j} \cdot \left( \frac{\partial f_j g_j - \partial g_j f_j}{g_j^2} \right) = \frac{\partial f_j g_j - \partial g_j f_j}{f_j g_j} = \frac{\partial f_j}{f_j} - \frac{\partial g_j}{g_j}
  \end{align*}
\end{proof}

\subsection{Computing the likelihood variety}

The ideal \( (\frac{\partial \ell_u}{\partial \theta_1}, \dots, \frac{\partial \ell_u}{\partial \theta_d}) \) is generated by \emph{rational} functions. Let's find another set of generators that consists of only polynomials. We introduce unknowns \( z = z_1,...,z_m \) where \( z_i \) represents \( f_i^{-1} = \frac{1}{f_i} \). So, we have two polynomial rings \( \mathbb Q[\theta] \) and \( \mathbb Q[\theta, z] \); clearly 
\begin{align*}
  \mathbb Q[\theta] \xhookrightarrow{} \mathbb Q[\theta, z].
\end{align*}
Consider the ideal \( J_u \) generated by \( d + m \) polynomials in \( \mathbb Q [\theta, z] \)
\begin{align*}
  J_u \coloneqq \left(
    \sum^m_{i=1}u_i z_i\frac{\partial f_i}{\partial \theta_1}, 
    \dots,
    \sum^m_{i=1}u_i z_i\frac{\partial f_i}{\partial \theta_d}, 
    z_1 f_1 - 1, \dots , z_m f_m - 1 
  \right).
\end{align*}
A point \( (\theta, z) \in \mathbb C^{d+m} \) lies in the variety \( V(J_u) \) if and only if
\begin{enumerate}
  \item \( \theta \) is a solution to the critical equations,
  \item \( f_i(\theta) \neq 0 \), and
  \item \( z_i = f_i^{-1}(\theta) \).
\end{enumerate}
 Next, we compute the \textbf{elimination ideal} of \( J_u \) in \( \mathbb Q[\theta] \), that is 
\begin{align*}
  I_u \coloneqq J_u \cap \mathbb Q[\theta]
\end{align*}
We call \( I_u \) the \textbf{likelihood ideal} of the model \( \mathbf f \) with respect to the data \( u \). A point \( \theta \in \mathbb C^d \) with \( f_i(\theta) \neq 0 \) lies in \( V(I_u) \) if and only if \( \theta \) is solution to the critical equations. \textbf{Thus, \( V(I_u) \) is the likelihood variety.}

\begin{remark}[Algorithm]
  \(  \)
\begin{enumerate}
  \item Compute the likelihood ideal: \( I_u = J_u \cap \mathbb Q [\theta] \)
  \item Compute \( V(I_u) \) (for example by computing a Gröbner basis).
  \item Compute \( S = V(I_u) \cap \mathbf f^{-1}(\Delta) \), where \( \Delta \) is the \( (m-1) \)-dimensional probability simplex.
  \item For each \( \theta \in S \) check if \( \mathbf f(\theta) \) is a local maxima (for example by examining the Hessian matrix).
\end{enumerate}
\end{remark}


\subsection{Maximum likelihood degree}
An important question for computational statistics is this:
\begin{mdframed}
  \begin{center}
    \textbf{{What happens to the estimate \( \hat \theta \) when we vary \( u \)?}}
  \end{center}
\end{mdframed}

\begin{defi}[Algebraic model]
  We say a model \( \mathbf f \) is \textbf{algebraic} if all the \( f_i \) are polynomials or rational functions.
\end{defi}

\begin{prop}[\( \hat \theta \) is an algebraic function of the data \( u \)]
  The maximum likelihood estimate \( \hat \theta \) is an algebraic function of the data \( u \) if \( \mathbf f \) is algebraic. That is, \( \hat \theta_i \) is a zero of a polynomial of the following form 
  \begin{align*}
    a_r(u) x^r + a_{r - 1}(u)x^{r-1} + ... + a_i(u) x + a_0(u),
  \end{align*}
  where each \( a_i \in \mathbb Q[u] \).
\end{prop}

Without loss of generality, we can assume that the polynomial is an \emph{irreducible element} of \( \mathbb Q[u, x] \). This means that \textbf{the discriminant is a nonzero polynomial in \( \mathbb Q[u] \).} 

\begin{defi}[Generic]
  We say that \( u \in \mathbb R^m \) is \textbf{generic} if no discriminant vanishes at \( u \) for all \( i=1,...,m \). Hence, there exist no multiple roots in any field extension (see Wikipedia, section \emph{Zero discriminant}). The generic vectors are dense in \( \mathbb R^m \).
\end{defi}

\begin{defi}[Maximum likelihood degree]
  The \textbf{maximum likelihood degree} or \textbf{ML degree} of an algebraic statistical model is the \emph{number of solutions to the critical equations} for generic data point \( u \in \mathbb R^m \).
\end{defi}


\end{document}

\section{Dimension Theory}

\begin{defi}[Affine Hilbert function]
  Let \( I \subset k[x_1,...,x_n] \) be an ideal. The \textbf{affine Hilbert function} of \( I \) is defined to be
  \begin{align*}
    \mathrm{aHF} _{R/I}:
    s \mapsto \mathrm{dim}\left( R_{\leq s} / I_{\leq s} \right).
  \end{align*}
\end{defi}

\begin{remark}[Finite dimensional vector space]
  Note that \( R_{\leq s} / I_{\leq s} \) is a subspace of the \( k \)-vector space \( R_{\leq s} \), the latter is a \emph{finite-dimensional} vector space since there exist \( {n + s  \choose s} \) monomials of degree \( \leq s \); these monomials form a \emph{basis}. So both the vector space and the subspace are finite dimensional and we can compute \( \mathrm{dim}\left( R_{\leq s} / I_{\leq s} \right) = \mathrm{dim}\left( R_{\leq s} \right) - \mathrm{dim}(I_{\leq s} ) \).
\end{remark}

For a monomial ideal we have an alternative interpretation of the affine Hilbert function: it counts the number of monomials not in the ideal.

\begin{prop}[Affine Hilbert function of monomial ideals]
  Let \( I \) be a monomial ideal. Then \( \mathrm{aHF}_{R / I}(s) \) is equivalent to the map 
  \begin{align*}
    \mathrm{aHF} _{R/I}:
    s \mapsto \text{  counts the number of monomial of degree \( \leq s \) not in \( I \)}.
  \end{align*}
\end{prop}

\begin{remark}
  If \( I \) is a monomial ideal, we know for sufficiently large \( s \) the above function can be represented by a polynomial, which we call the \textbf{Hilbert polynomial} \( \mathrm{aHP}_{R / I} \). Moreover, this polynomial is of degree \( \mathrm{dim}(V(I)) \), where by definition \( \mathrm{dim}(V(I)) \) is defined as the dimension of the largest coordinate subspace in \( V(I) \).
\end{remark}


\begin{prop}[Reduction to monomial ideals]
  For any graded order and any ideal \( I \), we have \( \mathrm{aHF}_{R / I} = \mathrm{aHF}_{R / (\mathrm{LT}(I))} \).
\end{prop}

This allows us to define the Hilbert polynomial  for arbitrary ideals. Just pick any graded order and define \( \mathrm{aHP} \) to be the polynomial representing \( \mathrm{aHF}_{R / \mathrm{LT}(I)} \).
\begin{align*}
  \mathrm{aHF}_{R / I} \coloneqq \mathrm{aHF}_{R / \mathrm{LT}(I)} = C(\mathrm{LT(I)}) = \text{Hilbert polynomial of \( \mathrm{LT}(I) \)}
\end{align*}

\begin{defi}[Affine Hilbert polynomial]
  Let \( I \) be an ideal in \( k[x_1,...,x_n] \). For sufficiently large \( s \), the polynomial \( \mathrm{aHP}_{R / I} \) that equals \( \mathrm{aHF}_{R / I} \) is called the \textbf{affine Hilbert polynomial}.
\end{defi}

As previously stated, the degree of the affine Hilbert polynomial equals the dimension of \( V(I) \) if \( I \) is a monomial ideal. 

\begin{defi}[Dimension of a variety]
  The \textbf{dimension of a variety} \( V \subset k^n \) is the degree of the affine Hilbert polynomial \( \mathrm{aHP}_{R / I(V)} \).
\end{defi}

We gave a purely algebraic description of the dimension of a variety:
\begin{align*}
  \text{dimension of a variety } = \text{ degree of a polynomial}
\end{align*}

\begin{remark}[Warning]
  Let \( V \) be any variety with \( V = V(I) \) for some ideal \( I \). Then the degree of the Hilbert polynomial of \( I \) need not be equal to the dimension of \( V \). This only holds for algebraically closed fields (if \( k = \bar k \), then the Nullstellensatz holds and \( I(V(I)) = \sqrt I \)).
  \begin{align*}
    I \text{ such that } V = V(I) \nRightarrow \mathrm{dim}(I) = \mathrm{dim}(V)
  \end{align*}
\end{remark}

\begin{prop}[Characterization of zero dimensional varieties]
  Let \( V \subset k[x_1,...,x_n] \) be a nonempty affine variety. Then 
  \begin{align*}
    |V| < \infty \iff \mathrm{dim}(V) = 0 .
  \end{align*}
\end{prop}

\begin{proof}
  If \( V  \) is empty, then the dimension is not defined. So assume \( V \neq \emptyset \).

  \begin{itemize}
    \item \( \implies \): Assume that \( V = \left\{ v_1,...,v_k \right\} \subset \mathbb R^n \). For each \( i=1,...,n \) we define the polynomial 
    \begin{align*}
      f_i(x) = (x_i - v_{1i})(x_i -  v_{2i}) \cdots (x_i - v_{ki}) \in I(V).
    \end{align*}
    Observe that \( \mathrm{LT}(f_i) = x_i^k \) for any graded order. So \( (\mathrm{LT}(I(V))) \) contains \( x_1^k, \dots , x_n^k \).

    By definition, 
    \begin{align*}
      \mathrm{dim}(V) = \mathrm{deg}(\mathrm{aHP}_{R / I(V)}) =  \mathrm{deg}(\mathrm{aHP}_{R / \mathrm{LT}(I(V))}).
    \end{align*}
    The degree of the Hilbert polynomial of a monomial ideal \( J \) equals the dimension of \( V(J) \) where the dimension of \( V(J) \) is defined to be the dimension of the largest coordinate subspace in \( V(J) \). Thus, by setting \( J = \mathrm{LT}(I(V)) \), we obtain
    \begin{align*}
      \mathrm{deg}(\mathrm{aHP}_{R / \mathrm{LT}(I(V))}) = \mathrm{dim}(V(\mathrm{LT}(I(V)))).
    \end{align*}
    Since \( \mathrm{LT}(I(V)) \) contains \( x_1^k, \dots , x_n^k  \), its vanishing ideal consists of points with \( x_1 = ... = x_n = 0 \). Hence, \( V(\mathrm{LT}(I(V))) = \left\{ 0 \right\} \). Clearly, \( \mathrm{dim}(\left\{ 0 \right\}) = 0\) (since any coordinate subspace of \( \left\{ 0 \right\} \) is of dimension \( 0 \)).

    \item \( \impliedby \): Let \( V \) be of dimension \( 0 \). Hence, the Hilbert polynomial of \( I(V) \) is a constant for sufficiently large \( s \). This means 
    \begin{align*}
      \mathrm{dim}(k[x_1,...,x_n]_{\leq s} / I(V)_{\leq s}) = C.
    \end{align*}
    Let \( s \geq C \). Then for any \( i=1, ..., n \) the set of vectors \( x_i^{\{0, \dots, s\}} \) is linearly dependent in \( k[x_1,...,x_n]_{\leq s} / I(V)_{\leq s} \). So, define the polynomial \( f_i \) to be
    \begin{align*}
      0 \neq f_i \coloneqq \sum_{k=0}^s \alpha_k x_i^k \in I(V)_{\leq s}.
    \end{align*}    
    Since this holds for any \( s \geq C \), \( f_i \neq 0 \) in \( I(V) \). Hence, \( f_i \in I(V) \) has only finitely many roots (since it is nonzero); also \( f_i \) vanishes on \( V \). Thus, \( V \) has only finitely elements \( y \in V \) with different coordinates \( y_i \). Since \( i \) was chosen arbitrarily, \( V \) is finite.
  \end{itemize}
\end{proof}

\section{Algebra Primer}

Here \( \mathbb{N} = \left\{ 0,1,2,3, ... \right\} \).

\begin{defi}[Monomial]
  An expression of the form \(  \mathbf x^{\mathbf u} = x_1^{u_1} x_2^{u_2} \dots x_n^{u_n} \) is called a \textbf{monomial} where \( \mathbf u \in \mathbb N^n \).
\end{defi}

\begin{defi}[Variety]
  Let \( S \subset k[\mathbf x] \) be a subset of polynomials. The \textbf{variety} of \( S \) is defined as
  \begin{align*}
    V(S) = \left\{ \mathbf a \in k \mid f(\mathbf a) = 0, \; \forall f \in S \right\}
  \end{align*}
\end{defi}

\begin{defi}[Vanishing ideal]
  Let \( W \subset k^n \). We define
  \begin{align*}
    I(W) = \left\{ f \in k[\mathbf x] \mid f(\mathbf w) = 0 \; \forall \mathbf w \in W \right\}.
  \end{align*}
\end{defi}

In algebraic statistics we often encounter varieties presented as \emph{\textbf{parametric sets}}
\begin{align*}
  V(S) = \left\{ \phi(t) : t \in k^d \right\} = \phi(k^d).
\end{align*}

\begin{eg}[Twisted cube]
  Consider \( \phi: \mathbb{C} \to \mathbb{C}^3: t \mapsto (t,t^2,t^3) \). We have 
  \begin{align*}
    V(\left\{ x^2 - y, x^3 - z \right\}) = \phi(k^d).
  \end{align*}
  \includegraphics*[width=0.5\textwidth]{assets/Twisted_cubic_curve.png}
\end{eg}


\begin{eg}[Binomial model]
  Consider the map \( \phi: \mathbb{C} \to \mathbb{C}^{n+1} \) given by \( \phi_i(t) = {n \choose i} t^{i}(1 - t)^{n- i} \) or alternatively
  \begin{align*}
    \phi: t \mapsto \begin{bmatrix}
      {n \choose 0} (1-t)^n \\ {n \choose 1} t (1-t)^{n-1} \\ \vdots \\ {n \choose n-1} t^{n - 1} (1 - t) \\ {n \choose n} t^n
    \end{bmatrix}.
  \end{align*}
  The map \( \phi_i(t) \) computes the probability to observe \( i \)-many successes where a success occurs with probability \( t \). In other words, \( \phi(t) \) denotes the probability distribution of a binomial random variable \( X \sim \mathrm{Bin}(n,t) \).

  The image \( \phi([0,1]) \) consists of all possible probability distributions of a binomial random variable; it is a curve inside the probability simplex \( \Delta_{n} \subset \mathbb R^{n + 1} \).

  \begin{figure}[H]
    \centering
    % \tdplotsetmaincoords{60}{130}
    % \begin{tikzpicture}[scale=2, tdplot_main_coords]
    %   \draw[thick,->] (0,0,0) -- (2,0,0) node[anchor=north east]{$x$};
    %   \draw[thick,->] (0,0,0) -- (0,2,0) node[anchor=north west]{$y$};
    %   \draw[thick,->] (0,0,0) -- (0,0,2) node[anchor=south]{$z$};
    %   \draw (1,0,0) -- (0,1,0) -- (0,0,1) -- (1,0,0);
    % \end{tikzpicture}
    \includegraphics*[width=0.75\textwidth]{assets/binomial_curve_on_simplex.png}
    \caption{The red curves shows all binomial distributions in the binomial model.}
  \end{figure}
  The collection of all probability distributions \( \phi(t) \) for \( t \in [0,1] \) will later be called a \textbf{parametric statistical model}.
\end{eg}

\begin{thm}[Hilbert basis theorem]
  Every ideal \( I \subset k[\mathbf x] \) is finitely generated. 
\end{thm}

\begin{remark}[Implicitization problem]
  Given a rational map \( \phi: k^d \to k^n \), we want to find a generating set of the ideal \( I(\phi(k^d)) \). This problem is called the \textbf{implicitization problem}.
\end{remark}


\begin{prop}[Implicit binomial statistical model]
  The binomial model \( \mathcal{M}_t = \left\{ \mathbf p \mid \text{\( \mathbf p \) is a binomial distribution} \right\} \) consists of points \( \mathbf x \in \mathbb{R}^{n+ 1} \) inside \( \Delta_n \) that satisfy all \( 2 \times 2 \) minors of the matrix
  \begin{align*}
    \begin{bmatrix}
      \dfrac{x_0}{{n \choose 0}} & \dfrac{x_1}{{n \choose 1}} & \dfrac{x_2}{{n \choose 2}} & \dots & \dfrac{x_{n-1}}{{n \choose n-1}} \\[1.5em]
      \dfrac{x_1}{{n \choose 1}} & \dfrac{x_2}{{n \choose 2}} & \dfrac{x_3}{{n \choose 3}} & \dots & \dfrac{x_{n}}{{n \choose n}}
    \end{bmatrix}.
  \end{align*}
\end{prop}

\begin{proof}
  We will solve it later using implicitization techniques.
\end{proof}

\begin{eg}[Binomial model]
  For \( \phi_i(t) = {2 \choose i} t^{i}(1 - t)^{2 - i} \) we have that 
  \begin{align*}
    I(\phi(\mathbb C)) = (x + y + z  - 1, 4xz - y^2).
  \end{align*}
  So the two polynomials \( x + y + z - 1 \) and \( 4xz - y^2 \) solve the implicitization problem.
\end{eg}

\begin{thm}[Nullstellensatz]
  If \( k = \bar k \), then \( I(V(I)) = \sqrt I \) for all ideals \( I \subset k[\mathbf x] \).
\end{thm}

\begin{defi}[Zariski closure]
  Let \( W \subset k^n \) be a subset. The set \( V(I(W)) \) is called the \textbf{Zariski closure} of \( W \). It is the smallest algebraic variety that contains \( W \).
\end{defi}


\subsection{Elimination}

\marginnote{\small{Lec, 11/06/23}}

\begin{defi}[Elimination ideal]
Let \( I  \) be an ideal in \( k[x_1, \dots, x_n] \) and \( j \in \left\{ 1, \dots, n \right\} \). The \( j \)-th elimination ideal of \( I \) is 
\begin{align*}
  I_j = I \cap k[x_{j+1}, x_{j+2}, \dots, x_n] \subset k[x_{j+1}, \dots, x_n].
\end{align*}
\end{defi}


\begin{prop}[Vanishing ideal of projection]
  Let \( V \subset k^{r + s} \) be a variety. Let \( \pi: k^{r + s} \to k^{r} \) be the projection. Then, the vanishing ideal of \( \pi(V) \) is the ideal \( I(\pi(V)) = I \cap k[x_1, \dots, x_r] \).
\end{prop}

\begin{proof}
  No proof.
\end{proof}

\begin{lemma}[Image of a variety under projection]\label{lem:image-of-variety-under-projection}
  Let \( I_j \) be the \( j \)-th elimination ideal of \( I \subset k[x_1, \dots, x_n] \) and \( \pi_j : k^n \to k^{n - j} \). Then \(   \pi_j(V(I)) \subset V(I_j)
  \).
\end{lemma}
\begin{proof}
  Let \( (a_{j+1}, \dots, a_n) \in \pi_j(V(I)) \); hence there exists \( \mathbf{a} \) such that \( \mathbf a = (a_1, \dots, a_n) \in V(I) \). Let \( f \in I_j \subset k[x_{j+1}, \dots, x_n] \). This \( f \) is also in \( I \subset k[x_1, \dots, x_n] \). So \( f(a_1, ... , a_n) = 0 \) where \( f \in I \). Hence, \( \pi_j (V(I)) \subset V(I_j) \).
\end{proof}

\begin{lemma}\label{lem:closure-theorem-lemma}
  If \( k = \bar k \), then \( I(\pi_j(V(I))) \subset \sqrt{I_j} \).
\end{lemma}

\begin{proof}
  Let Hilbert's Nullstellensatz (HNS) do all the work!
  Take some \( f \in I(\pi_j(V(I))) \subset k [x_{j+1}, \dots, n] \). It means \( f(a_{j+1}, \dots, a_n) = 0 \) for all \( a = (a_1, \dots, a_n) \in V(I) \) or in other words \( f(a_1, \dots, a_n)  = 0 \) in \( k[x_1, \dots, x_n] \) for all \( a \in V(I) \). Hence \( f \in I(V(I)) = \sqrt I \) by HNS. Hence \( f^m \in I \) for some \( m \). Hence \( f^m \) still cannot contain variables \( x_1, \dots, x_j \). Hence \( f^m \in k[x_{j+1}, \dots, x_n] = I_j \). Thus, \( f \in \sqrt{I_j} \).
\end{proof}

\begin{mdframed}
\begin{thm}[Closure Theorem]
If \( k = \bar k \), then \(\overline{\pi_j(V(I))} =  V(I_j)  \).
\end{thm}
\end{mdframed}

\begin{proof}
  Use Lemma \ref{lem:image-of-variety-under-projection}: \( \overline{\pi_j(V(I))} \subset \overline{V(I_j)} = V(I_j) \).

  Use Lemma \ref{lem:closure-theorem-lemma} and applying \( V(\cdot) \) yields \( V(\sqrt I_j) = V(I_j)  \subset V(I(\pi_j(V(I))) = \overline{\pi_j(V(I))}) \).
\end{proof}


\begin{eg}
Let same \( I = (xy - 1) \subset k[x,y] \). Then \( V(I) = \left\{ (x,\frac{1}{x}) \mid x \neq 0 \right\} \). Then \( \pi_1(V(I)) = k \setminus \left\{ 0 \right\} \subset k = V(I_1) \) where \( I_1 = \left\{ 0 \right\} \).
\end{eg}

\begin{eg}
Let \( k = \mathbb{R} \). Let \( I = (x^2 + y^2) \subset \mathbb R [x,y]\). Then \( V(I) = \left\{ (0,0) \right\} \). Then \( \pi_1(V(I)) = \left\{ 0 \right\} \subset \mathbb R = V(\left\{ 0 \right\}) \) with \( I_1 = \left\{ 0 \right\} \).
\end{eg}


\marginnote{\small{Lec, 11/07/23}}

\begin{defi}
  A \textbf{$j$-th elimination ordering} is a term ordering where any monomial with \( x_i, i \leq j \) indeterminate appearing to any nonzero power is larger than every monomial only in \( x_i, i > j \) indeterminates.
\end{defi}

\begin{mdframed}
\begin{thm}[Elimination Theorem]\label{thm:elimination-theorem}
Let \( I \) be an ideal in \( k[x_1, ..., x_n] \) with Gröbner basis \( G \) with respect to a \( j \)-th elimination ordering. Then \( G_j = G \cap k[x_{j+1}, ... , x_n] \) is a Gröbner basis of \( I_j \) with respect to \( \prec \).
\end{thm}
\end{mdframed}

\begin{proof}
  To show that \( G_j \) is a Gröbner basis of \( I_j \) we need to show 
  \begin{itemize}
    \item \( G_j \subset I_j \): this follows because \( G \subset I \);
    \item \(  \mathrm{in}_\prec(G_j) \subset \mathrm{in}_\prec (I_j) \): this is clear;
    \item \(  \mathrm{in}_\prec(G_j) \supset \mathrm{in}_\prec (I_j) \): Let \( f \in I_j \). Then \( f \in I \). Therefore, there exist some \( g \in G \) such that \( \mathrm{in}_\prec(g)  \) divides \( \mathrm{in}_\prec(f) \). Note that \( \mathrm{in}_\prec(f) \) only contains variables \( x_{j+1}, ..., x_n \). Hence, \( \mathrm{in}_\prec(g) \) does not contain variables \( x_1, ..., x_j \). Since we have an elimination order, all the other terms of \( g \) do not contain \( x_1, ..., x_j \), as well. So \( g \in G_j \). Thus, \( f \) is generated by some element in \( \mathrm{in}_\prec(G_j) \), namely \( g \).
  \end{itemize}
\end{proof}

\begin{cor}[Lexicographic ordering]
An ideal \( I \subset k [x_1,...,x_n] \), \( G \) a Gröbner basis of \( I \) with respect to lex \( x_1 > x_2 > ... > x_n \) implies that \( G_j = G \cap k[x_{j+1}, ..., x_n] \) is a Gröbner basis of \( I_j \)  for all \( j = 1 , ... , n \).
\end{cor}

\begin{eg}[Compute the closed image of a variety under projection]
  Let \( k = \mathbb{C} \).
  Define \( f(x,y) = xy^3 - x^2 \) and \( g(x,y) = x^3y^2 - y \). Consider the ideal \( I = (f,g) \) and its variety \( V(I) \). What is \( \overline{\pi_1(V(I))} \)?

  \begin{itemize}
    \item Compute a Gröbner basis of \( I \) with respect to lexicographic ordering \( x \prec y \). We have  \( S(f,g) = y^2 - x^4 \). The standard residue is \( h (x,y) = y^2 - x^4 \). So, we add it to \( G \). Computing the \( S \)-polynomial between \( f,h \) and \( g,h \) yields: \( k = f - xy(y^2 - x^4) = x^5y - x^2 \) and\( l = g - x^3(y^2 - x^4) = x^7 - y \). Next, we need to compute the standard residue of \( k \) and \( l \). Repeating this process yields the unique reduced Gröbner basis \( G = \left\{ y - x^7, x^5y - x^2 \right\} \). Now if we plug \( x^7 = y \) into \( x^5y - x^2 \) we obtain \( G = \left\{ y - x^7, x^{12} - x^2 \right\} \).
    \item Compute \( G_1 = \left\{ x^{12} - x^2 \right\} \).
    \item Compute \( \overline{\pi_1(V(I))} = V(I_1) = V(G_1) = V(x^{12} - x^2) = \left\{ 0 \right\} \cup \left\{ \xi \mid \xi^{10} = 1 \right\} \subset \mathbb{C}\).
  \end{itemize}
\end{eg}

\subsection{Implicitization}

The \emph{implicitization problem} is concerned with the following problem: Find the generators of the vanishing ideal of an image of a polynomial map; more precisely, given \( \varphi: k^d \to k^r \) we want to compute the generators of \( I(\varphi(k^d)) \).

\begin{eg}[Twisted cube]
  Let \( \varphi: k \to k^3, t \mapsto (t,t^2,t^3)  \). Then \( I(\varphi(k)) = (y - x^2, z - x^3) \).
\end{eg}

\begin{eg}[Binomial model with two trials]
  Let \( X \sim \mathrm{Bin}(2, \theta) \). Then, the probability distribution is \( \phi(\theta) = p_\theta = \begin{bmatrix}(1 - \theta)^2 & 2 \theta(1 - \theta) & \theta^2 \end{bmatrix} \) for \( \theta \in [0,1] \). What relations are satisfied by the components of \( p_\theta \)? This is answered by implicitization. We will later see that \( p_0p_2 - \frac{1}{4}p_1^2 \) holds.
\end{eg}

\begin{eg}[Warning: Image of a polynomial function need not be a variety]
Consider \( \phi: \mathbb{C}^2 \to \mathbb{C}^2, (\theta_1, \theta_2) \mapsto (\theta_1, \theta_1\theta_2) \). Then, \( \mathrm{Im}(\phi) = \left\{ (0,0) \right\} \cup \left\{ (a,b) \in \mathbb{C}^2 \mid a \neq 0 \right\}\), which is not a variety. Note that the image is dense, i.e. \( \overline{\mathrm{Im}\phi} = \mathbb{C}^2 \). 
\end{eg}

\begin{remark}[Constructible set]
  It can be shown that if we have \( k = \bar k \) and a polynomial function \( \phi: k^d \to k^r \), then there exist a finite sequence of varieties \( W_1 \supset W_2 \supset ... \supset W_s \) in \( k^r \) such that \( \phi({k^d}) = W_1 \setminus(W_2 \setminus ... \setminus W_s) \). This is called a \emph{constructible} set. 

  In the example above, \( \phi(\mathbb{C}^2) = \mathbb{C}^2 \setminus (\{ (0,a) \mid a \in \mathbb{C} \} \setminus \left\{ (0,0) \right\}) \).

  Note if \( k \neq \bar k \), then \( \phi: \mathbb{R} \to \mathbb{R}, \phi(\theta) = \theta^2 \) and \( \mathrm{Im}(\theta) = [0, \infty) \). However, it is true that \( \phi(\mathbb{R}^d) \) is always a semialgebraic set (given by polynomial inequalities).

\end{remark}


\begin{mdframed}  
\begin{thm}[Implicitization theorem]
Fix \( k = \bar k \). Let \( \phi = (\phi_1, \dots, \phi_r): k^d \to k^r \) be a polynomial. Define the ideal \( I = (y_1 -  \phi_1, y_2 - \phi_2,\dots, y_r - \phi_r) \subset k [x_1, ..., x_d, y_1, ..., y_r] \). Then, \( \overline{\mathrm{image}(\phi)} = V(I_d).  \)
\end{thm}
\end{mdframed}

\begin{proof}
  We have \(   V(I) = \left\{ (x_1,...,x_d, y_1, ..., y_r) \mid \phi(x_1,...,x_d) = (y_1,..., xy_r) \right\}
  \).
This is the graph of \( \phi \). Then
\begin{align*}
  \pi_d(V) &= \left\{ (x_1, ..., x_r) \mid \phi_1(t_1,...,t_d) = x_1, ..., \phi_r(t_1,...,t_d) = x_r \right\} \\
  &= \left\{ (\phi_1(t_1,...,t_d), ... , \phi_r(t_1,..., t_d)) \mid (t_1,..., t_d) \in k^d \right\} = \phi(k^d).
\end{align*}
In words, the image of the parametrization is the projection of its graph. By closure theorem, we can compute the image of a projection by the elimination ideal; hence \( V(I_d) = \overline{\pi_d (V(I))} = \overline{\phi(k^d)} \).
\end{proof}

\begin{eg}[Binomial model with two trials]
  Let \( X \sim B(2, \theta) \) whose distribution is parametrized by \( \phi: k \to k^3 \), \( \phi(\theta) = \begin{bmatrix}
    (1 - \theta)^2 & 2\theta(1 - \theta) & \theta^2
  \end{bmatrix} \). What relations are satisfied by the components of \( \phi(\theta) \)?

  Define the ideal \( I = (p_0 - (1 - \theta)^2, p_1 - 2\theta(1 - \theta), p_2 - \theta^2) \subset \mathbb{C}[\theta, p_0, p_1, p_2] \). By the Implicitization theorem, we want to compute \( I \cap k[p_0,p_1,p_2] \). Using the Elimination theorem, we can compute this elimination ideal with Gröbner basis. Consider the lexicographic ordering \( y_3 \prec y_2 \prec y_1 \prec \theta \). Using Macaulay2 we obtain the reduced Gröbner basis 
  \begin{align*}
    G &= \left\{ 2 \theta - p_1 - 2p_2, p_0 + p_1 + p_2 - 1, p_1^2 + 4 p_1 p_2 + 4p_2^2 - 4p_2 \right\} \\
    G_{\mathbf p} &= \left\{ p_0 + p_1 + p_2 - 1, p_1^2 + 4 p_1 p_2 + 4p_2^2 - 4p_2 \right\} 
  \end{align*}
  Thus, by the Elimination theorem we obtain 
  \begin{align*}
    V(I_1) = V(G_1) = V(p_0 + p_1 + p_2 - 1,  p_1^2 + 4 p_1 p_2 + 4p_2^2 - 4p_2).
  \end{align*}
  Note that the first equation implies \( -p_1 -p_2 + 1 = p_0 \) and the second implies \( p_1^2 + 4 p_1 p_2 + 4p_2^2 - 4p_2 = p_1^2 - 4p_2 (-p_1 - p_2 + 1) \); hence we plug the first into the second equation to obtain the \emph{Hardy Weinberg equations}
  \begin{align*}
    V( p_0 + p_1 + p_2 - 1, p_0 p_2 - \frac{1}{4}p_1^2).
  \end{align*}
\end{eg}

\begin{eg}[General binomial model]
  In general for \( X \sim \mathrm{Bin}(n, \theta) \) this model is parametrized by \( \phi = (\phi_0, \phi_1 , \dots, \phi_n) \), where \( \phi_i = P(X = i) = {n \choose i}\theta^{i} (1 - \theta)^{n - i} \). The implicit equations of \( \phi(k) \) are \( p_0 + p_1 + ... + p_n = 1 \) and the \( 2 \times 2 \)-minors of the Hankel matrix 
\begin{align*}
  \begin{bmatrix}
    p_0 / {n \choose 0} & p_1  / {n \choose 1} & ... & p_{n-1}  / {n \choose n-1} \\
    p_1 / {n \choose 1} & p_2  / {n \choose 2} & ... & p_{n}  / {n \choose n}
  \end{bmatrix}
\end{align*}
For \( n = 2 \), we obtain \( \mathrm{det}\begin{bmatrix}
  p_0 & p_1 / 2 \\ p_1 / 2& p_2
\end{bmatrix} = p_0 p_2 - \frac{1}{4}p_1^2\).
\end{eg}



\subsection{Primary decomposition}

\begin{defi}[Prime and reducible ideals]
  An ideal \( I \) is \textbf{prime} if \( ab \in I \) implies that \( a \in I \) or \( b \in I \). An ideal \( I \) is called \textbf{reducible} if \( I = I_1 \cap I_2 \) for proper ideals \( I_1 , I_2 \supsetneq I  \).
\end{defi}

\begin{defi}[Reducible variety]
  A variety \( V \) is called \textbf{reducible} if \( V' \cup V'' = V \) for proper varieties \( V, V' \subsetneq V \).
\end{defi}

\begin{prop}
  A variety is irreducible if its associated ideal is prime.
\end{prop}


\begin{prop}
  Every variety can be decomposed \emph{uniquely} into finitely many irreducible varieties: \( V = V_1 \cup ... \cup V_m \).
\end{prop}


\begin{prop}
  Every ideal in a Noetherian ring can be decomposed into finitely many irreducible ideals: \( I = I_1 \cap ... \cap I_m \).
\end{prop}

\begin{defi}[Primary ideal and associated prime ideal]
  An ideal \( I \) is called \( J \)-\textbf{primary} if \( xy \in I \) and \( x \notin I \) implies that \( y \in J \coloneqq \sqrt{I} \). We call \( J = \sqrt I \) the \textbf{associated prime} of the primary ideal \( I \).
\end{defi}


\begin{prop}[Some facts in a Noetherian ring]\(  \)
  \begin{itemize}
    \item Every prime ideal is irreducible.
    \item Not every irreducible ideal is prime.
    \item Every irreducible ideal is primary.
    \item The intersection of prime ideals need not be prime (e.g. take \( (2) \) and \( (3) \)).
    \item The intersection of \(J\)-primary ideals is \( J \)-primary.
  \end{itemize}
\end{prop}

\begin{defi}[Primary decomposition]
  Let \( I \subset k[x] \) be an ideal. A \textbf{primary decomposition} of \( I \) is a representation of \( I = \bigcap_{i=1}^r Q_{i} \) as an intersection of finitely many primary ideals \( Q_i \). The representation is \textbf{irredundant} if all \( \sqrt Q_i \) are distinct and no \( Q_i \supset \bigcap_{j \neq i} Q_j  \) exists.
\end{defi}

\begin{cor}
  A primary decomposition always exists in a Noetherian ring.
\end{cor}

\begin{defi}[Associated primes]
  Let \( I \) be decomposable such that \( I = Q_1 \cap ... \cap Q_m \) is irredundant. Then \( \sqrt Q_i \) are called the \textbf{associated primes} \( \mathrm{Ass}(I) \) of \( I \).
\end{defi}

\begin{prop}[1st uniqueness]
  Let \( I \subset k[x] \). Then, the associated primes of \( I \) are uniquely determined by \( I \); thus the associated primes are independent of the specific primary decomposition of \( I \). In other words, different irredundant primary decomposition yield the same associated primes.
\end{prop}

\begin{prop}[2nd uniqueness]
  The minimal ideals in \( \mathrm{Ass}(I) \) are called \textbf{minimal primes}. Primary ideals associated to the minimal ideals are uniquely determined by \( I \). In other words, 
  \begin{align*}
    I = \bigcap Q_i = \bigcap Q'_j, \quad \sqrt Q_k = \sqrt {Q'_{k'}} \; \text{is minimal for some \( k \in \mathbb N \)} \implies Q_k = Q_{k'}'.
  \end{align*}
\end{prop}

\begin{prop}
  Minimal primes are the smallest prime ideals containing \( I \).
\end{prop}

\begin{defi}
  \textbf{Embedded primes} are associated primes that are not minimal.
\end{defi}

\begin{mdframed}
  Associated primes are unique. \\
  The primary ideals associated to minimal primes are unique. \\
  The primary ideals associated to embedded primes need \emph{not} be unique.
\end{mdframed}

\begin{prop}[Irreducible composition and minimal primes]
  The irreducible composition of a variety \( V(I) \) is the union of \( \bigcup_{i} V(P_i) \) where \( P_i \) are the minimal primes of \( I \).
\end{prop}


\begin{defi}[Quotient ideal and saturation]
  Let \( I,J \) be ideals in \( K[x] \). The \textbf{quotient ideal} is the ideal \(     I : J = \left\{ f \in K[x] : fg \in I\; \forall g \in J \right\} \). The \textbf{saturation} of \( I \) by \( J \) is defined as \(     I : J^\infty = \bigcup^\infty_{k=1} I : J^k
  \).
\end{defi}

\begin{prop}
  We have 
  \begin{align*}
    I(V \setminus W) = I(V) : I(W).
  \end{align*}
  Moreover,
  \begin{align*}
    \overline{V(I) \setminus V(J)} = V(I : J^\infty).
  \end{align*}
\end{prop}

\begin{defi}[Binomial ideal]
  A \textbf{binomial} is a polynomial of the form \( \mathbf x^\alpha - \lambda \mathbf x^\beta \) in \( k[\mathbf x] \). A \textbf{binomial ideal} is an ideal generated by binomials.
\end{defi}